%%%%%%%%%%%%%%%%%%%%%%%%%%%%%%%%%%%%%%%%%
% University/School Laboratory Report
% LaTeX Template
% Version 3.1 (25/3/14)
%
% This template has been downloaded from:
% http://www.LaTeXTemplates.com
%
% Original author:
% Linux and Unix Users Group at Virginia Tech Wiki 
% (https://vtluug.org/wiki/Example_LaTeX_chem_lab_report)
%
% License:
% CC BY-NC-SA 3.0 (http://creativecommons.org/licenses/by-nc-sa/3.0/)
%
%%%%%%%%%%%%%%%%%%%%%%%%%%%%%%%%%%%%%%%%%

%----------------------------------------------------------------------------------------
%	PACKAGES AND DOCUMENT CONFIGURATIONS
%----------------------------------------------------------------------------------------

\documentclass[11pt,a4paper]{article}
\usepackage[utf8]{inputenc}
\usepackage[english,polish]{babel}
%\usepackage[T1]{fontenc}
\usepackage{polski}
\usepackage{amsfonts}
\usepackage[left=3cm,right=3cm,top=3cm,bottom=3cm]{geometry}
\usepackage{siunitx} % Provides the \SI{}{} and \si{} command for typesetting SI units
\usepackage{graphicx} % Required for the inclusion of images
\usepackage{natbib} % Required to change bibliography style to APA
\usepackage{amsmath} % Required for some math elements
\usepackage{epstopdf}
\usepackage[colorlinks=false,urlcolor=blue,citecolor=green]{hyperref}
\usepackage{fancyhdr}
\usepackage{lastpage}
\usepackage{array}
\usepackage{hhline}
\usepackage{svg}
\usepackage{multirow}
\usepackage{enumerate}%[I], numerki, [(a)]
\usepackage{float}

%Figure numbering
\usepackage{chngcntr}
\counterwithin{figure}{section}
\counterwithin{equation}{section}

\newcommand*{\captionsource}[2]{%
  \caption[{#1}]{%
    #1%
    \\\hspace{\linewidth}%
    \textbf{Źródło:} #2%
  }%
}

\AtBeginDocument{
	\renewcommand{\tablename}{Tabela}
	\renewcommand{\figurename}{Rys.}
}

%tabelki
\usepackage{tabularx}
\newcolumntype{A}{>{\centering\arraybackslash}X}
\newcolumntype{B}{>{\centering\arraybackslash} m{0.4\textwidth} }

\setlength\parindent{0pt} % Removes all indentation from paragraphs

\renewcommand{\labelenumi}{\alph{enumi}.} % Make numbering in the enumerate environment by letter rather than number (e.g. section 6)

%\usepackage{times} % Uncomment to use the Times New Roman font

\title{\textbf{Lewitacja magnetyczna}} % Title

\author{Marcin Kowalczyk \\ Mateusz Ługowski \\ Karolina Szmyd} % Author name

\date{22 maj, 2017} % Date for the report

\begin{document}

\maketitle % Insert the title, author and date

\begin{center}
\begin{tabular}{l r}
Data wykonania: & 10 maj, 2017 \\ % Date the experiment was performed
Przedmiot: & Laboratorium Problemowe \\
Prowadzący: & Dawid Knapik % Instructor/supervisor
\end{tabular}
\end{center}

\section{Wstęp}
\label{sec:wstep}
Celem projektu było zaprojektowanie systemu sterowania dla problemu lewitacji magnetycznej. Opiera się ona na powstanie siły w ferromagnetyku umieszczonym z zewnętrznym polu magnetycznym. Siła grawitacji działająca na metalową kulkę ma być równoważona przez oddziaływania magnetyczne. Wytwarzane pole magnetyczne zależy od prądu płynącego przez cewkę. Układ umożliwia pomiar położenia obiektu, sterowanie napięciem podawanym na cewkę generującą pole magnetyczne oraz umożliwia współpracę z urządzeniami sterującymi. Do pomiaru położenia lewitującej sfery służy układ optyczny. Im wyżej znajduje się sfera, tym bardziej zasłania źródło światła, a więc mniej światła pada na fototranzystor. Mierzony jest również prąc płynący przez cewkę za pomocą napięcia na rezystorze. Zakłada się, że ruch kulki odbywa się tylko w osi pionowej.

Do symulacji, obliczeń numerycznych i generowania kodu wykorzystano programy \textit{Matlab} i \textit{Simulink}. Wygenerowany kod jest wykonywany w czasie rzeczywistym. Możliwa jest również zmiana parametrów układu sterowania (np. pozycji zadanej) w czasie pracy układu. Mierzone wielkości są również wyświetlane na bieżąco w programie \textit{Simulink}.

\section{Model matematyczny}
\label{sec:modelmatematyczny}
Model matematyczny układu został wyprowadzony z wykorzystaniem funkcji Lagrange'a. Jest ona różnicą energii kinetycznej i potencjalnej.
\begin{equation}
T=\frac{m\dot x^2}{2}+\frac{L(x)\dot q^2}{2}+\frac{1}{2}\int\limits_0^t R\dot q^2dt+mgx+qu
\label{eq:flagrange}
\end{equation}
\noindent Gdzie:\newline
\(x\) jest odległością sfery od elektromagnesu.\newline
\(q\) jest ładunkiem przepływającym przez cewkę.\newline
\(I=\dot q\) jest prądem płynącym w cewce.\newline
\(m\) jest masą sfery.\newline
\(R\) jest rezystancją cewki.\newline
\(L(x)\) jest indukcyjnością w zależności od położenia sfery.\newline
\(g\) jest przyspieszeniem ziemskim.\newline
\(u\) jest napięciem sterującym.

Funkcje \(x(t)\) i \(q(t)\) spełniają następujący układ równań:
\begin{equation}
\begin{aligned}
\frac{d}{dt}\frac{dT}{d\dot x}-\frac{dT}{dx}&=0\\
\frac{d}{dt}\frac{dT}{d\dot q}-\frac{dT}{dq}&=0
\end{aligned}
\label{eq:lagrange}
\end{equation}

Na podstawie równań \eqref{eq:flagrange} i \eqref{eq:lagrange} otrzymuje się następujące równania opisujące dynamikę systemu:
\begin{equation}
\begin{aligned}
\frac{d^2x}{dt^2}&=\frac{1}{2m}\frac{dL}{dx}I^2+g\\
\frac{dI}{dt}&=\frac{1}{L}(-\frac{dL}{dx}\frac{dx}{dt}I-RI+u)
\end{aligned}
\label{eq:dynamic}
\end{equation}

Siła działająca na sferę wyrażona jest następującą zależnością:
\begin{equation}
F(x,I)=\frac{1}{2}\frac{dL}{dx}I^2
\end{equation}

Przyjęto, że wartość induktancji w zależności od odległości od cewki ma następującą postać:
\begin{equation}
L(x)=L_0+L_1e^{-ax}
\end{equation}
\noindent Gdzie:\newline
\(L_0\) jest indukcyjnością cewki bez ferromagnetycznej sfery.

Założono, że układ, którego wyjściem jest prąd cewki, a wejściem napięcie podawane na cewkę, jest obiektem inercyjnym pierwszego rzędu.
\begin{equation}
\frac{dI}{dt}=-\frac{1}{T}I+\frac{k}{T}(u+u_c)
\label{eq:coil_current}
\end{equation}
\noindent Gdzie:\newline
\(T\) jest stałą czasową.\newline
\(k\) jest współczynnikiem wzmocnienia.\newline
\(u\) jest napięciem sterującym.\newline
\(u_c\) jest stałym napięciem na cewce.

Wyliczono równania stanu opisujące działanie układu lewitacji magnetycznej.
\begin{equation}
\begin{aligned}
\frac{dx_1}{dt}&=x_2\\
\frac{dx_2}{dt}&=\frac{1}{2m}\frac{dL(x_1)}{dx_1}x_3^2+g+z_1(t)\\
\frac{dx_3}{dt}&=-\frac{1}{T}x_3+\frac{k}{T}(u(t)+u_c+z_2(t))
\end{aligned}
\label{eq:ss}
\end{equation}
\noindent Gdzie:\newline


\section{Identyfikacja}
\label{sec:identyfikacja}


%\section{Wyniki}
%Na rysunku \ref{fig:rezim_slizgowy} przedstawiono zjawisko reżimu ślizgowego dla układu z regulatorem dwupołożeniowym.
%
%\begin{figure}[h]
%	\centering
%	\includegraphics[width=4in]{rezim_slizgowy.eps}
%	\caption{Portret fazowy z widocznym reżimem ślizgowym.}
%	\label{fig:rezim_slizgowy}
%\end{figure}
%
%Na rysunku \ref{fig:cykl_graniczny_pf} przedstawiono portret fazowy z widocznym stabilnym cyklem granicznym w układzie z regulatorem dwupołożeniowym z histerezą. Na rysunku \ref{fig:cykl_graniczny_ster} widoczne jest przykładowe sterowanie dla układu ze stabilnym cyklem granicznym, a na rysunku \ref{fig:cykl_graniczny_zs} przykładowy przebieg czasowy zmiennych stanu dla tego układu.
%
%\begin{figure}[h]
%	\centering
%	\includegraphics[width=4in]{cykl_graniczny_pf.eps}
%	\caption{Portret fazowy z widocznym cyklem granicznym.}
%	\label{fig:cykl_graniczny_pf}
%\end{figure}
%
%\begin{figure}[h]
%	\centering
%	\includegraphics[width=4in]{cykl_graniczny_ster.eps}
%	\caption{Przebieg czasowy sterowania dla układu z cyklem granicznym.}
%	\label{fig:cykl_graniczny_ster}
%\end{figure}
%
%\begin{figure}[H]
%	\centering
%	\includegraphics[width=4in]{cykl_graniczny_zs.eps}
%	\caption{Przebiegi czasowe zmiennych stanu dla układu z cyklem granicznym.}
%	\label{fig:cykl_graniczny_zs}
%\end{figure}
%
%Na rysunku \ref{fig:tropolozeniowy_hist_pf} przedstawiono portrety fazowe dla układu z regulatorem trójpołożeniowym. Widoczne są dwie proste przełączeń.
%
%\begin{figure}[h]
%	\centering
%	\includegraphics[width=4in]{trojpolozeniowy_hist_pf.eps}
%	\caption{Portret fazowy z układu z regulatorem trójpołożeniowym.}
%	\label{fig:tropolozeniowy_hist_pf}
%\end{figure}

\bibliographystyle{apalike}

\bibliography{sample}

%----------------------------------------------------------------------------------------


\end{document}