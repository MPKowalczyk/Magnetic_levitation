%%%%%%%%%%%%%%%%%%%%%%%%%%%%%%%%%%%%%%%%%
% University/School Laboratory Report
% LaTeX Template
% Version 3.1 (25/3/14)
%
% This template has been downloaded from:
% http://www.LaTeXTemplates.com
%
% Original author:
% Linux and Unix Users Group at Virginia Tech Wiki 
% (https://vtluug.org/wiki/Example_LaTeX_chem_lab_report)
%
% License:
% CC BY-NC-SA 3.0 (http://creativecommons.org/licenses/by-nc-sa/3.0/)
%
%%%%%%%%%%%%%%%%%%%%%%%%%%%%%%%%%%%%%%%%%

%----------------------------------------------------------------------------------------
%	PACKAGES AND DOCUMENT CONFIGURATIONS
%----------------------------------------------------------------------------------------

\documentclass[11pt,a4paper]{article}
\usepackage[utf8]{inputenc}
\usepackage[english,polish]{babel}
%\usepackage[T1]{fontenc}
\usepackage{polski}
\usepackage{amsfonts}
\usepackage[left=3cm,right=3cm,top=3cm,bottom=3cm]{geometry}
\usepackage{siunitx} % Provides the \SI{}{} and \si{} command for typesetting SI units
\usepackage{graphicx} % Required for the inclusion of images
\usepackage{natbib} % Required to change bibliography style to APA
\usepackage{amsmath} % Required for some math elements
\usepackage{epstopdf}
\usepackage[colorlinks=false,urlcolor=blue,citecolor=green]{hyperref}
\usepackage{fancyhdr}
\usepackage{lastpage}
\usepackage{array}
\usepackage{hhline}
\usepackage{svg}
\usepackage{multirow}
\usepackage{enumerate}%[I], numerki, [(a)]
\usepackage{float}

%Figure numbering
\usepackage{chngcntr}
\counterwithin{figure}{section}
\counterwithin{equation}{section}

\newcommand*{\captionsource}[2]{%
  \caption[{#1}]{%
    #1%
    \\\hspace{\linewidth}%
    \textbf{Źródło:} #2%
  }%
}

\AtBeginDocument{
	\renewcommand{\tablename}{Tabela}
	\renewcommand{\figurename}{Rys.}
}

%tabelki
\usepackage{tabularx}
\newcolumntype{A}{>{\centering\arraybackslash}X}
\newcolumntype{B}{>{\centering\arraybackslash} m{0.4\textwidth} }

%\setlength\parindent{0pt} % Removes all indentation from paragraphs

\renewcommand{\labelenumi}{\alph{enumi}.} % Make numbering in the enumerate environment by letter rather than number (e.g. section 6)

%\usepackage{times} % Uncomment to use the Times New Roman font

\title{\textbf{Lewitacja magnetyczna}} % Title

\author{Marcin Kowalczyk \\ Mateusz Ługowski \\ Karolina Szmyd} % Author name

\date{22 marzec, 2017} % Date for the report

\begin{document}

\maketitle % Insert the title, author and date

\begin{center}
\begin{tabular}{l r}
Data wykonania: & 22 marzec, 2017 \\ % Date the experiment was performed
Przedmiot: & Laboratorium Problemowe \\
Prowadzący: & Dawid Knapik % Instructor/supervisor
\end{tabular}
\end{center}

\section{Wstęp}
Celem ćwiczenia było zbadanie działania układu składającego się z regulatora przekaźnikowego i liniowego obiektu sterowania drugiego rzędu. Zapoznano się z dwoma zjawiskami występującymi w układach nieliniowych: reżimem ślizgowym i cyklem granicznym. Reżimy ślizgowe występują w układach przełączanych i o zmiennej strukturze. Cykle graniczne spotyka się różnych systemach nieliniowych. Wykorzystano portrety fazowe do zobrazowania wspomnianych zjawisk.

%\section{Wyniki}
%Na rysunku \ref{fig:rezim_slizgowy} przedstawiono zjawisko reżimu ślizgowego dla układu z regulatorem dwupołożeniowym.
%
%\begin{figure}[h]
%	\centering
%	\includegraphics[width=4in]{rezim_slizgowy.eps}
%	\caption{Portret fazowy z widocznym reżimem ślizgowym.}
%	\label{fig:rezim_slizgowy}
%\end{figure}
%
%Na rysunku \ref{fig:cykl_graniczny_pf} przedstawiono portret fazowy z widocznym stabilnym cyklem granicznym w układzie z regulatorem dwupołożeniowym z histerezą. Na rysunku \ref{fig:cykl_graniczny_ster} widoczne jest przykładowe sterowanie dla układu ze stabilnym cyklem granicznym, a na rysunku \ref{fig:cykl_graniczny_zs} przykładowy przebieg czasowy zmiennych stanu dla tego układu.
%
%\begin{figure}[h]
%	\centering
%	\includegraphics[width=4in]{cykl_graniczny_pf.eps}
%	\caption{Portret fazowy z widocznym cyklem granicznym.}
%	\label{fig:cykl_graniczny_pf}
%\end{figure}
%
%\begin{figure}[h]
%	\centering
%	\includegraphics[width=4in]{cykl_graniczny_ster.eps}
%	\caption{Przebieg czasowy sterowania dla układu z cyklem granicznym.}
%	\label{fig:cykl_graniczny_ster}
%\end{figure}
%
%\begin{figure}[H]
%	\centering
%	\includegraphics[width=4in]{cykl_graniczny_zs.eps}
%	\caption{Przebiegi czasowe zmiennych stanu dla układu z cyklem granicznym.}
%	\label{fig:cykl_graniczny_zs}
%\end{figure}
%
%Na rysunku \ref{fig:tropolozeniowy_hist_pf} przedstawiono portrety fazowe dla układu z regulatorem trójpołożeniowym. Widoczne są dwie proste przełączeń.
%
%\begin{figure}[h]
%	\centering
%	\includegraphics[width=4in]{trojpolozeniowy_hist_pf.eps}
%	\caption{Portret fazowy z układu z regulatorem trójpołożeniowym.}
%	\label{fig:tropolozeniowy_hist_pf}
%\end{figure}

\bibliographystyle{apalike}

\bibliography{sample}

%----------------------------------------------------------------------------------------


\end{document}